% \documentclass[lipt]{article}

\documentclass[UTF8]{article}
% 中文包
\usepackage{CTEX}


\usepackage{graphicx}
\usepackage{float}
\usepackage{apacite}


% set no indent for each paragrah by default 
\setlength\parindent{0pt}



\title{Python 笔记}
\author{MarkSCQ}
\date{2021, June}


\begin{document}

\maketitle
  
\tableofcontents

\newpage

\section{Python知识点回顾及补充}
\section{基本语法}
Python 基础补充。语法,到一些特性特点。\newline
Python 变量作用域
*args,**kwargs

\subsection{Python 数据结构的用法还有特点}
List, Tuple, Set, Dictionary等

\subsection{异常处理}

try except else finally \newline
assert,断言。 From W3C School, the assert keyword let you test if a condition in your code return  True, if not, the program will raise an AssertionError.\newline

raise 触发异常

\subsection{OOP}

继承 , 多继承 \newline
函数重写 \newline
类属性  \_\_private\_attrs \newline
类函数:public/private \newline
重载(overload)\newline
重写(overwrite)\newline
覆盖(overrode)\newline

\begin{itemize}
    \item \_\_init\_\_
    \item \_\_new\_\_
\end{itemize}


\subsection{正则表达式}
知识空白,re基础库

\subsection{标准库}
os, sys, re, datetime

\subsection{IO处理}
text, csv, excel, json, 文件写入和读,追加/覆盖...

\section{Django}

补充 Django 基础知识。目前对于Django的一些使用,前后端数据交互,Model,

% \subsection{}

\section{Scrapy}
初步探索Scrapy。

\section{Python 面试}
收集一些Django岗面试题,或者python岗面试题。
[:: - 1]
\end{document}